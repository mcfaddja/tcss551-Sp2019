\documentclass{article}[12pt]
\setlength{\textheight}{8.75in}
\setlength{\topmargin}{-0.5in}
\setlength{\oddsidemargin}{-.25in}
\setlength{\evensidemargin}{0in}
\setlength{\textwidth}{7in}
\usepackage{amsfonts, amsmath, amsthm, amssymb, mathrsfs, graphicx, fancyhdr, cancel, latexsym, multicol, setspace}

%\usepackage{savesym}
%\usepackage{amsmath}
%\savesymbol{iint}
%\usepackage{txfonts}
%\restoresymbol{TXF}{iint}

\pagestyle{fancy}
\lhead{Jonathan McFadden \\ Spring 2019}
\rhead{TCSS 551 \\ Term Project}
\headsep = 22pt 
\headheight = 15pt


\doublespacing

% BEGIN PRE-AMBLE


% Setup equation numbering 
\numberwithin{equation}{subsection} 

%Equation Numbering Shortcut Commands
\newcommand{\numbch}[1]{\setcounter{section}{#1} \setcounter{equation}{0}}
\newcommand{\numbpr}[1]{\setcounter{subsection}{#1} \setcounter{equation}{0}}
\newcommand{\note}{\textbf{NOTE:  }}

%Formatting shortcut commands
\newcommand{\chap}[1]{\begin{center}\begin{Large}\textbf{\underline{#1}}\end{Large}\end{center}}
\newcommand{\prob}[1]{\textbf{\underline{Problem #1):}}}
\newcommand{\sol}[1]{\textbf{\underline{Solution #1):}}}
\newcommand{\MMA}{\emph{Mathematica }}

%Text Shortcut Command
\newcommand{\s}[1]{\emph{Side #1}}

% Math shortcut commands
\newcommand{\pd}[2]{\frac{\partial #1}{\partial #2}}
\newcommand{\pdn}[3]{\frac{\partial^{#1} #2}{\partial #3^{#1}}}
\newcommand{\infint}{\int_{-\infty}^\infty}
\newcommand{\infiint}{\iint_{-\infty}^\infty}
\newcommand{\infiiint}{\iiint_{-\infty}^\infty}
\newcommand{\dint}[2]{\int_{#1}^{#2}}
\newcommand{\dd}[1]{\textrm{d#1}}
\newcommand{\ddd}[1]{\textrm{d}#1}
\renewcommand{\Re}{\mathbb{R}}

%Math Text
\newcommand{\csch}{\text{ csch}}

%Physics Shortcut Commands
\newcommand{\h}{\mathcal{H}}
\newcommand{\Z}{\mathcal{Z}}


% END PRE-AMBLE




\begin{document}

\begin{center}
\begin{Large} \textbf{\underline{Forest Cover Analysis}} \end{Large}
\end{center}

\begin{center}
\begin{large} \emph{TCSS-551 Term Project} \end{large}
\end{center}

\begin{center} Jonathan McFadden \end{center}


\vspace{1.5in}



\begin{flushleft}


\begin{large} \textbf{\underline{Introduction}} \end{large}

The goal of this project is twofold.  First, we seek to create an accurate big-data/machine-learning model to predict which type of trees will grow best in a certain area based on the conditions of that area.  Finally, we seek to determine which features of a given area are most important for determine which type of tree will grow best there.  To do this, we have chosen three different methods:

\begin{itemize}
\item Linear/Logistic Regression
\item Decision Trees
\item Random Forest
\end{itemize}

We chose to try a Linear/Logistic regression first because they're provide a simple and low computational cost way to see a relationship between the features and outcomes of the provided data does indeed exist.  This method will be unlikely produce predictions with good accuracy due to its simplicity, however should these methods show a relationship or relationships within the provided data, that would be a positive indicator to proceed to more complicated and compute intensive algorithms.


\vspace{0.5in}

\begin{large} \textbf{\underline{METHOD A} - Linear/Logistic Regression} \end{large}

As described above, the initial method we tried were Linear/Logistic Regressions.  We tried three variations on this method.  First, we ran a simple linear regression.  This gave a poor result, but \emph{did} indicate that there was a relation between the features and outcomes in the data.  This linear model gave an $R^2$ value of $0.4214$ on the test data and $0.4006$ on the training data.  This is odd as, one would expect this value to be higher for the training data than the test data.  This trend also continued when comparing via MSE\footnote{mean squared error}, as the training data had an MSE of $2.3952$ while the test data had an MSE of $2.3257$.  \newline

We followed this basic Linear Regression model with two Logistic Regression models; one based on fitting a binary problem for each class (\emph{OVR}) and the other based on a multinomial fitting (\emph{MULTNOM}).  The \emph{OVR} Logistic Regression had a mean error of $0.6711$ and an MSE of $2.8735$ on the test data.  The \emph{MULTNOM} Logistic Regression did slightly better on mean error with $0.6653$ and slightly worse on the MSE with $3.2866$ on the test data.  Interestingly, the \emph{OVR} Logistic Regression had a better MSE on the test data than on the training data ($2.9991$) while the \emph{MULTNOM} Logistic Regression had a better mean error on the test data than the training data ($0.6721$). \newline

While this method did establish that \emph{there is} a relationship between the features and outcomes in the data, it does not have very good accuracy.  More importantly, it does not tell us anything about which features are most important for determining which type of tree will do best in a certain area.  However, now that we know there is a relationship within the data, we can move on to more complicated and computationally intense methods to determine the importance of the various features.


\vspace{0.5in}

\begin{large} \textbf{\underline{METHOD B} - Decision Trees} \end{large}




























































\end{flushleft}

\end{document}
